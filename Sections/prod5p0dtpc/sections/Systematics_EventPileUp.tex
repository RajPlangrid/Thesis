\subsection{Event Pileup per Bunch}
\label{sec:Systematics_EventPileUp}
The pileup\footnote{Coincidental neutrino interactions in the same bunch.} 
event contribution in the \p0d CC inclusive samples 
should be proportional to the experimental beam power. 
From the fact that the estimated beam power used in the simulated MC 
was not the exact same as in each of the runs, we expect 
to have a different pileup effect between data and MC.
%Taken in to account the areal density of the \p0d 
%and the short time of each beam bunch, this is a small effect 
%as our study shows. 
Here we have estimated these effect difference on our 
final data to MC ratio of each run. \\

To estimate the event pileup effect we have assumed 
that there is 
an equal probability for an interaction to occur in each beam bunches. 
We can then evaluate this probability from the number of active bunches 
and the number of total bunches available for each Run both for Data and MC.
These probabilities squared are our estimations to have two 
interaction in the same bunch, i.e. the event pileup rate.
These rates were found to be very small and are summarized 
in Tables \ref{tab:PileUpEventsWaterIn} and \ref{tab:PileUpEventsWaterOut}. 
%In Run 1 extracted rates are 
%0.0000003 for Data and 0.0000005 for MC. 
%Very similar trend between a Data and MC were found in Run 2 
%i.e. 0.0000003 and 0.0000008 respectively.\\

\begin{table}[h]
\centering
\begin{tabular}{ccc}\toprule
 & & Event Pileup Fractional error \\
%\hline
\cline{3-3}
Run 1 & & $+$0.0000002 \\
Run 2 & & $+$0.000000001 \\
Run 4 & & $-$0.0000021 \\
\bottomrule
\end{tabular} 
\caption{
Event pileup fractional error on the Data/MC ratios 
as been extracted for the Water-in periods: Run 1, Run 2 and Run 4.
}
\label{tab:PileUpEventsWaterIn}
\end{table}

\begin{table}[h]
\centering
\begin{tabular}{ccc}\toprule
 & & Event Pileup Fractional error \\
%\hline
\cline{3-3}
Run 2 & & $-$0.0000008 \\
Run 3 & & $-$0.0000017 \\
Run 4 & & $-$0.0000037 \\
\bottomrule
\end{tabular} 
\caption{
Event pileup fractional error on the Data/MC ratios 
as been extracted for the Water-out periods: Run 2, Run 3 and Run 4.
}
\label{tab:PileUpEventsWaterOut}
\end{table}
 
%As in previous systematic estimations, 
%this study used Global Reconstruction as opposed to the Tracker to \p0d 
%matching algorithm used for the current CC inclusive analysis selection. 
%We do not expect coincidence rates to change appreciably from switching matching algorithms, 
%especially when the new algorithm has also demonstrated 
%to have high reconstruction efficiency. 
The systematic error on the Data to MC ratios 
for a specific run can be estimated with the use of the pileup event rates 
of both Data and MC. These were calculated for all run periods 
separately and yielded very small systematic errors 
(see Tables \ref{tab:PileUpEventsWaterIn} 
and \ref{tab:PileUpEventsWaterOut}).\\
The fact that the observed uncertainty are negligible have 
caused us not to include them in the find systematics calculation.

%estimated taking into account the spill probability factors 
%to have two interactions in two different bunches or in the same one.
%
%This, combined with the fact that the observed uncertainty is negligible, 
%allows us to quote $- 0.00005\%$, the larger of the two fractional errors 
%from Table \ref{tab:PileUpEvents}, as the event pile-up systematic.
