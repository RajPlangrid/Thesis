\subsubsection{Track Breaking}
\label{sec:Systematics_TrackBreaking}

{\color{red} This could potentially be a source of a large difference between MC
and data. Barcelona completed a rather involved study. Our HM track selection
allows the track to start in any detector. This means that a broken track might
cause an incorrect HM selection and then fail our fiducial cut. To determine the
level of this effect, we should use a single track topology that starts in the
p0d and goes through the TPCs and determine how often it breaks. Of this broken
track sample, we would run our HM selection on it to see how often the second
track segment is selected instead of the first. We may use a sample of through
going tracks or a reconstructed version of the p0d recon efficiency sample
described above.

The other option would make this simpler. We could alter our cuts so that
instead of having our HM selection choose from THE highest momentum track in the
ND280 detector, we just choose out of the highest momentum track in the \p0d with
a TPC1 piece. This effectively eliminates a track multiplicity systematic as
having multiple tracks from the same vertex is a physics systematic, not a
detector one. However, this would actually change the sampled physics in our
selection.

Upon further discussion with Anthony, he dug up some numbers related to this.
The track breaking percentage in data is 96+/-.2 percent and in monte carlo is
97.5+/-0.1 percent...A clear difference. We need to find a way to propagate this
to a systematic error in our analysis. As I said, this effect applies to us only
if the second track segment was selected as the highest momentum track instead
of the first. So essentially we need the number of tracks in data / MC during
our selection method that failed our fiducial cut due to track breaking. Since
we don't want to create a whole new sample, what I think makes sense is to ask
Anthony for a sample of his "`broken"' tracks, and run our analysis on them. All
we are looking for are the number of times in a broken track sample that the
second track segment has a higher momentum than the first.

}
