\subsubsection{Beam Nu-E and Neutral Current Background [Do not edit!!]}
\label{sec:Systematics_NC}

In our CC Inclusive sample, we have a fraction of events determined by monte carlo as either Neutral Current interactions or electron type neutrino interactions. The neutral current interactions have an uncertainty of 30\% as determined by the NIWG group. This uncertainty is lower depending on the type of neutral current interaction, but we choose to conservatively assign a flat uncertainty for all neutral current interactions in our charged current sample. By weighting the NC cross section uncertainty with the fraction of NC events in our CC selection, we calculate the NC bacground systematic uncertainty. \\
Similarly, we refer to the \p0d Nu-E technical note to calculate the systematic from the Nu-E content of the beam. The \p0d reports a ratio of $1.17 \pm 0.10(stat.) \pm 0.19(sys.)$ between Data and MC for the beam Nu-E content. Instead of correcting the Nu-E background to an imprecise central value of 1.17, we take instead the full range of the ratio as given by its added errors. We say that our Nu-E background can vary from $1.17+0.10+0.19 = 1.47$ to $1.17-0.10-0.19 = 0.88$ in its Data to MC ratio. When weighted with the fraction of Nu-E events in our CC selection, the mentioned range yields the final systematic for beam Nu-E contamination.\\
There are some caveats to the described approach. Firstly, the Out of '\p0d Fiducial Volume' background contains some NC and Nu-E events so the systematics will have some small amount of correlation. This correlation will be accounted for in an update of this document and is expected to be quite small. Secondly, the Nu-E systematic was calculated strictly from the uncertainty in the beam content, not also from the Nu-E cross section uncertainties. The effect of cross section uncertainty should not be very high either given the Quasi-elastic like nature of the contaminant events. Finally, the Nu-E component has not been reweighted following the same prescription as the Nu-Mu component but will be soon.\\
For Run 1, according to our MC studies, 
the NC and Nu-E content are 1.6 \% and 0.72 \% respectively. 
This yields an NC systematic of $\pm 0.5 \% $ 
and a Nu-E systematic of (rounding conservatively) +0.4\% -0.1\%. 
For Run 2, we find a NC component of 1.56\% and Nu-E component of 0.68\%, yielding very similar systematics of $\pm 0.5\%$ and +0.4\% -0.1\%.
