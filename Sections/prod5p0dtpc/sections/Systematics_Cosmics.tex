\subsection{Cosmics}
\label{sec:Systematics_Cosmics}

To estimate the cosmic contamination of our data samples 
we have defined the time period before the first bunch in each spill 
as a sideband sample, 
for each T2K run. \\

\begin{table}[h]
\centering
\begin{tabular}{ccccc}
%\hline
\toprule
   & & Run 1& & Run 2\\
%\hline
\cline{3-3}\cline{5-5} 
Cosmic candidate tracks& & 2 & & 0 \\
\hline 
%\cline{3-3}\cline{5-5} 
Total cosmic scanned time & & 2.429$\times 10^9$ ns & & 2.811$\times 10^9$ ns \\
\hline 
%Cosmic rate& &8.233$\cdot 10^{-10}$ track/ns& &3.557$\cdot 10^{-10}$ tracks/ns \\
%\hline 
%\cline{3-3}\cline{5-5} 
Total analysis time & & 2.159$\times 10^9$ ns & & 3.331$\times 10^9$ ns \\
\hline 
%\cline{3-3}\cline{5-5} 
Expected cosmic contamination & & 2 tracks & & 0 tracks \\
\bottomrule
\end{tabular} 
\caption{Cosmic contamination rates for the two run periods.}
\label{tab:CosmicsStudy}
\end{table}

These sideband samples were used to scan for tracks that would 
pass all of our selection cuts. 
The number of tracks found \footnote{In case no tracks were found, 
an upper limit of one track was used.} in each run is documented 
in Table \ref{tab:CosmicsStudy}. 
This number was then divided by the 'Total cosmic scanned time', 
i.e. number of spills $\times$ time period before first bunch in ns, 
to extract the rate of contaminated tracks per time unit for each run. 
To find the 'Expected cosmic contamination' number in each run, 
we multiplied the above rate by the 'Total analysis time' 
(which is the number of spills times number of bunches 
in ns). 
The results of these multiplications are rounded up 
to evaluate the number of contaminated cosmic tracks in our 
selected samples. 
Table \ref{tab:CosmicSystematics} 
summarizes all the above findings and calculations. \\

We adopt a conservative approach which assumes that when 
a contaminated track is
present in a bunch it is tagged as the muon candidate. 
We then recalculate the Data/MC ratios without these tracks and extract 
a systematic upper limit for this contamination source.

\begin{table}[h]
\centering
\begin{tabular}{ccccc}
%\hline
\toprule
   & & Run 1& & Run 2\\
\cline{3-3}\cline{5-5} 
%\cline{3-3}\cline{5-5} 
Cosmic systematics & & $-$0.00031 & & 0 \\
%\hline
\bottomrule
\end{tabular} 
\caption{Cosmic systematic values for the two run periods.}
\label{tab:CosmicSystematics}
\end{table}

We note that the study in this section has been completed with the use 
of tracks reconstructed by the Global Reconstruction package. 
The CC inclusive selection outlined in Section \ref{sec:InclusiveAnalysis} 
was performed with a Tracker to \p0d matching algorithm. 
Though these two reconstruction methods are not the exact same, 
they are very similar. In addition, from the efficiency study performed 
using cosmics (Sec. \ref{sec:CosmicsEfficiency})  
we find that the new matching algorithm maintains a high efficiency 
of reconstructing cosmics and other lengthy tracks. 
Since the overall contribution of the cosmics contamination is negligible, 
we do not expect that switching to the Tracker to \p0d matching method 
will change this systematic uncertainty. 
We adopted the conservative approach and estimate our cosmic 
systematic contribution to be $\pm0.00031$.

