\section{Introduction}
\label{sec:Introduction}

Almost 85 years ago, Wolfgang Pauli proposed to the physics community the existence of a neutral, weakly interacting particle that would solve the baffling problem of the continuous $\beta$ decay energy spectrum. In the many decades that have passed, not only has the $\beta$ spectrum puzzle been solved, but the story of this little particle, dubbed the ``neutrino", has grown into an entire science. The study of neutrino properties and their interactions with matter have consumed many millions of man-hours and motivated the construction of enormous technological structures all over the world. 

In 1960, Davis and Bahcall found a deficit in the number of expected neutrinos from the sun; a problem with the rather surprising solution that particular flavors of neutrinos could somehow \emph{disappear} and \emph{reappear} over time and distance. Evidence for this phenomenon, called neutrino oscillation, was later discovered by Super Kamiokande and published in the most cited particle physics paper of all time. The Solar Neutrino Observatory (SNO) later confirmed that neutrino mixing in combination with the matter effect solved the solar neutrino problem. Neutrinos have also proved instrumental in probing the structure of a nucleon. Yet many mysteries remain unraveled. Neutrinos have mass, but how much mass? Are they Dirac or Majorana particles? Can neutrino oscillation violate charge-parity conservation and even further the understanding of the matter-antimatter asymmetry of our universe? These questions and more mean that the study of neutrinos will remain a fascinating and growing field.

The T2K experiment is one of a few long-baseline neutrino experiments designed to measure the parameters that govern neutrino oscillation. The far detector of T2K is the same Super Kamiokande responsible for the discovery of neutrino mixing. Super Kamiokande is a water Cherenkov detector, so measurement of oscillation parameters requires an accurate knowledge of the neutrino interaction rate on water. Also, neutrino nucleon interactions are the only way to measure the axial form factors of the nucleon current. With such motivations, the need for a neutrino cross section on water is clear.

In this thesis, we present a measurement of the $\nu_\mu$ induced charged current inclusive cross section on water using the Pi-Zero detector and the Tracker of the off-axis near detector complex in the T2K experiment. In a $\nu_\mu$ CC inclusive interaction, we expect a daughter muon to be produced. Using the uniquely designed water target of the P0D as a neutrino interaction target, and the TPC for its muon reconstruction, we select the daughter muon in the charged current event. The neutrino event rate in the P0D while it is drained of water is statistically subtracted from the neutrino event rate in the P0D while it is filled with water.

In Chapter \ref{sec:Theory}, we first discuss the formalism describing neutrino oscillation and then summarize the calculation of the three main neutrino-nucleon scattering processes in the 1~GeV regime: quasi-elastic, resonance production and deep inelastic scattering. Then, in Section \ref{sec:detectordescription}, we provide a description of the T2K experiment, focusing on the near detector complex where the cross section measurement is carried out. Section \ref{sec:FluxDetermination} discusses how the muon neutrino flux at the near detector is determined and Section \ref{sec:evsim} describes the neutrino interaction simulation and how T2K uses external data to constrain cross section parameters. Section \ref{sec:Reconstruction} explains the event reconstruction process. Then Section \ref{sec:fidmassvol} describes how the mass of water in the analysis fiducial volume is measured. Section \ref{sec:selection} describes the selection process used to identify candidate muon neutrino events and shows the resulting distributions. Section \ref{sec:xsec} uses the selected event rates and develops the statistical subtraction method used to extract the water-only cross section from two separate samples. Sections \ref{sec:fluxxsecsyst}, \ref{sec:detsys} and \ref{sec:detsysprop} explain how flux, cross section parameter and detector systematic uncertainties are measured and propagated into the water cross section measurement. Finally, Section \ref{sec:results} discusses the final measurement of the $\nu_\mu$ charged-current inclusive cross section on water and compares it to available global data. 

As with all modern neutrino experiments, T2K is a large collaboration. A collection of theorists, experimentalists, engineers and graduate students are intellectually responsible for the complete set of analyses performed using T2K apparatus. A lot of information in this thesis is extracted from publications, presentations, notes and technical documents written by these people. It is therefore important to highlight my personal intellectual contributions to this thesis. In close collaboration with small analysis groups at Colorado State University, Fort Collins and University of Colorado, Boulder, I developed the track matching algorithm in Section \ref{sec:matching}, the selection procedure in Section \ref{sec:selection}, and the methodology behind measuring detector systematic uncertainties in Section \ref{sec:detsys}. The propagation of flux and cross section model uncertainties in Section \ref{sec:fluxxsecsyst} was done primarily by myself with the help of a reweighting software package developed by T2K collaborators. The cross section extraction formalism in Section \ref{sec:xsec} and the propagation of detector systematic uncertainties and application of corrections in Section \ref{sec:detsysprop} were entirely my work. The introduction in each Section will also delineate work that is ours and work that is from other T2K subgroups. This thesis would not be possible without the massive, international, collaborative effort that it was my pleasure to be a part of.


